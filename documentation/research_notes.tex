\documentclass{article}

\usepackage{hyperref}
\usepackage{enumitem}

\usepackage[letterpaper,top=2cm,bottom=2cm,left=3cm,right=3cm,marginparwidth=1.75cm]{geometry}

\title{COSC 444/544 - Computer Vision}
\author{Project Part I, Research Notes}

\begin{document}
\maketitle

\section*{Research Papers}

\begin{enumerate}
\item Riley Eaton
  \begin{enumerate}[label*=\arabic*.]
  \item \textit{\href{https://typeset.io/papers/visual-object-tracking-using-adaptive-correlation-filters-1xuhtpe358}{Visual Object Tracking using Adaptive Correlation Filters}}
  \item \textit{\href{https://faculty.ucmerced.edu/mhyang/papers/cvpr13_benchmark.pdf}{Online Object Tracking: A Benchmark}}
  \end{enumerate}
\item Wanju Luo
  \begin{enumerate}[label*=\arabic*.]
  \item \textit{\href{paper_link_here}{Paper \#1 Name}}
  \item \textit{\href{paper_link_here}{Paper \#2 Name}}
  \end{enumerate}
\item Dichen Feng
  \begin{enumerate}[label*=\arabic*.]
  \item \textit{\href{paper_link_here}{Paper \#1 Name}}
  \item \textit{\href{paper_link_here}{Paper \#2 Name}}
  \end{enumerate}
\item Henry Pak
  \begin{enumerate}[label*=\arabic*.]
  \item \textit{\href{paper_link_here}{Paper \#1 Name}}
  \item \textit{\href{paper_link_here}{Paper \#2 Name}}
  \end{enumerate}
\item Santam
  \begin{enumerate}[label*=\arabic*.]
  \item \textit{\href{paper_link_here}{Paper \#1 Name}}
  \item \textit{\href{paper_link_here}{Paper \#2 Name}}
  \end{enumerate}
\end{enumerate}

\section*{Research Summary}

The following are summaries of the papers selected by each group member.

% -------------------------- RILEY EATON PAPER 1 --------------------------
\subsection*{1.1 \textit{Visual Object Tracking using Adaptive Correlation Filters}}

\hspace*{\parindent}\textbf{Reviewer:} Riley Eaton

\vspace{0.3cm}

\textbf{Summary:} 

\vspace{0.3cm}

\textbf{Key Points \& Concepts:}
\begin{itemize}
  \item \textbf{Template matching} is one of the simplest object tracking approaches. A single patch of an image, called a \emph{template}, is chosen. Then, in each new frame the template is slid across the entire image and a similarity measure is computer at each location. Whatever position has the highest similarity score is the new estimated position of the object. 
  \item \textbf{Convolution} involves flipping the kernel (or filter) and sliding it across the image. The dot product of the kernel and the image is computed at each location. This is similar to the similarity measure in template matching.
  \item \textbf{Correlation} by contrast does not flip the filter, and is usually used to measure similarity between images or image patches. This can be used to find how closely a test patch in a new frame matches a template patch from a previous frame.
  \item \textbf{Frequency Domain Computations} are used to speed up the computation of the correlation between the filter and the image. A Fast Fourier Transform (FFT) is used to convert the image and filter into the frequency domain, where the dot product is computed. The inverse FFT (iFFT) is then used to convert the result back to the spatial domain. This is computationally more efficient than the spatial domain computation, and speeds up matching for high frame rate applications.
\end{itemize}

\vspace{0.3cm}

% -------------------------- RILEY EATON PAPER 2 --------------------------
\subsection*{1.2 \textit{Online Object Tracking: A Benchmark}}

\hspace*{\parindent}\textbf{Reviewer:} Riley Eaton

\vspace{0.3cm}

\textbf{Summary:}

\vspace{0.3cm}

\textbf{Key Points \& Concepts:}
\begin{itemize}
  \item \textbf{Point} Description
\end{itemize}

% -------------------------- WANJU LUO PAPER 1 --------------------------
\subsection*{2.1 \textit{Paper Name}}

\hspace*{\parindent}\textbf{Reviewer:} Wanju Luo

\vspace{0.3cm}

\textbf{Summary:}

\vspace{0.3cm}

\textbf{Key Points \& Concepts:}
\begin{itemize}
  \item \textbf{Point} Description
\end{itemize}

% -------------------------- WANJU LUO PAPER 2 --------------------------
\subsection*{2.2 \textit{Paper Name}}

\hspace*{\parindent}\textbf{Reviewer:} Wanju Luo

\vspace{0.3cm}

\textbf{Summary:}

\vspace{0.3cm}

\textbf{Key Points \& Concepts:}
\begin{itemize}
  \item \textbf{Point} Description
\end{itemize}

% -------------------------- DICHEN FENG PAPER 1 --------------------------
\subsection*{3.1 \textit{Paper Name}}

\hspace*{\parindent}\textbf{Reviewer:} Dichen Feng

\vspace{0.3cm}

\textbf{Summary:}

\vspace{0.3cm}

\textbf{Key Points \& Concepts:}
\begin{itemize}
  \item \textbf{Point} Description
\end{itemize}

% -------------------------- DICHEN FENG PAPER 2 --------------------------
\subsection*{3.2 \textit{Paper Name}}

\hspace*{\parindent}\textbf{Reviewer:} Dichen Feng

\vspace{0.3cm}

\textbf{Summary:}

\vspace{0.3cm}

\textbf{Key Points \& Concepts:}
\begin{itemize}
  \item \textbf{Point} Description
\end{itemize}

% -------------------------- HENRY PAK PAPER 1 --------------------------
\subsection*{4.1 \textit{Paper Name}}

\hspace*{\parindent}\textbf{Reviewer:} Henry Pak

\vspace{0.3cm}

\textbf{Summary:}

\vspace{0.3cm}

\textbf{Key Points \& Concepts:}
\begin{itemize}
  \item \textbf{Point} Description
\end{itemize}

% -------------------------- HENRY PAK PAPER 2 --------------------------
\subsection*{4.2 \textit{Paper Name}}

\hspace*{\parindent}\textbf{Reviewer:} Henry Pak

\vspace{0.3cm}

\textbf{Summary:}

\vspace{0.3cm}

\textbf{Key Points \& Concepts:}
\begin{itemize}
  \item \textbf{Point} Description
\end{itemize}

% -------------------------- SANTAM PAPER 1 --------------------------
\subsection*{5.1 \textit{Paper Name}}

\hspace*{\parindent}\textbf{Reviewer:} Santam

\vspace{0.3cm}

\textbf{Summary:}

\vspace{0.3cm}

\textbf{Key Points \& Concepts:}
\begin{itemize}
  \item \textbf{Point} Description
\end{itemize}

% -------------------------- SANTAM PAPER 2 --------------------------
\subsection*{5.2 \textit{Paper Name}}

\hspace*{\parindent}\textbf{Reviewer:} Santam

\vspace{0.3cm}

\textbf{Summary:}

\vspace{0.3cm}

\textbf{Key Points \& Concepts:}
\begin{itemize}
  \item \textbf{Point} Description
\end{itemize}

\end{document}
