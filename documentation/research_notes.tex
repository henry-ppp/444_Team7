\documentclass{article}

\usepackage{hyperref}
\usepackage{enumitem}

\usepackage[letterpaper,top=2cm,bottom=2cm,left=3cm,right=3cm,marginparwidth=1.75cm]{geometry}

\title{COSC 444/544 - Computer Vision}
\author{Project Part I, Research Notes}

\begin{document}
\maketitle

\section*{Research Papers}

\begin{enumerate}
\item Riley Eaton
  \begin{enumerate}[label*=\arabic*.]
  \item \textit{\href{https://typeset.io/papers/visual-object-tracking-using-adaptive-correlation-filters-1xuhtpe358}{Visual Object Tracking using Adaptive Correlation Filters}}
  \item \textit{\href{https://faculty.ucmerced.edu/mhyang/papers/cvpr13_benchmark.pdf}{Online Object Tracking: A Benchmark}}
  \end{enumerate}
\item Wanju Luo
  \begin{enumerate}[label*=\arabic*.]
  \item \textit{\href{paper_link_here}{Paper \#1 Name}}
  \item \textit{\href{paper_link_here}{Paper \#2 Name}}
  \end{enumerate}
\item Dichen Feng
  \begin{enumerate}[label*=\arabic*.]
  \item \textit{\href{https://jst.org.in/index.php/pub/article/view/743/669}{Moving Object Tracking and Detection in Videos using MATLAB: A Review}}
  \item \textit{\href{https://www.semanticscholar.org/paper/Robust-visual-tracking-method-based-on-Harris-Hawks-Charef-Khodja-Abida/e945f79be12f7d64df3d5ef69256e2a0eaec1f03}{Robust visual tracking method based on Harris Hawks algorithm}}
  \end{enumerate}
\item Henry Pak
  \begin{enumerate}[label*=\arabic*.]
  \item \textit{\href{paper_link_here}{Paper \#1 Name}}
  \item \textit{\href{paper_link_here}{Paper \#2 Name}}
  \end{enumerate}
\item Santam
  \begin{enumerate}[label*=\arabic*.]
  \item \textit{\href{paper_link_here}{Paper \#1 Name}}
  \item \textit{\href{paper_link_here}{Paper \#2 Name}}
  \end{enumerate}
\end{enumerate}

\section*{Research Summary}

The following are summaries of the papers selected by each group member.

% -------------------------- RILEY EATON PAPER 1 --------------------------
\subsection*{1.1 \textit{Visual Object Tracking using Adaptive Correlation Filters}}

\hspace*{\parindent}\textbf{Reviewer:} Riley Eaton

\vspace{0.3cm}

\textbf{Summary:} 

\vspace{0.3cm}

\textbf{Key Points \& Concepts:}
\begin{itemize}
  \item \textbf{Template matching} is one of the simplest object tracking approaches. A single patch of an image, called a \emph{template}, is chosen. Then, in each new frame the template is slid across the entire image and a similarity measure is computer at each location. Whatever position has the highest similarity score is the new estimated position of the object. 
  \item \textbf{Convolution} involves flipping the kernel (or filter) and sliding it across the image. The dot product of the kernel and the image is computed at each location. This is similar to the similarity measure in template matching.
  \item \textbf{Correlation} by contrast does not flip the filter, and is usually used to measure similarity between images or image patches. This can be used to find how closely a test patch in a new frame matches a template patch from a previous frame.
  \item \textbf{Frequency Domain Computations} are used to speed up the computation of the correlation between the filter and the image. A Fast Fourier Transform (FFT) is used to convert the image and filter into the frequency domain, where the dot product is computed. The inverse FFT (iFFT) is then used to convert the result back to the spatial domain. This is computationally more efficient than the spatial domain computation, and speeds up matching for high frame rate applications.
\end{itemize}

\vspace{0.3cm}

% -------------------------- RILEY EATON PAPER 2 --------------------------
\subsection*{1.2 \textit{Online Object Tracking: A Benchmark}}

\hspace*{\parindent}\textbf{Reviewer:} Riley Eaton

\vspace{0.3cm}

\textbf{Summary:}

\vspace{0.3cm}

\textbf{Key Points \& Concepts:}
\begin{itemize}
  \item \textbf{Point} Description
\end{itemize}

% -------------------------- WANJU LUO PAPER 1 --------------------------
\subsection*{2.1 \textit{Paper Name}}

\hspace*{\parindent}\textbf{Reviewer:} Wanju Luo

\vspace{0.3cm}

\textbf{Summary:}

\vspace{0.3cm}

\textbf{Key Points \& Concepts:}
\begin{itemize}
  \item \textbf{Point} Description
\end{itemize}

% -------------------------- WANJU LUO PAPER 2 --------------------------
\subsection*{2.2 \textit{Paper Name}}

\hspace*{\parindent}\textbf{Reviewer:} Wanju Luo

\vspace{0.3cm}

\textbf{Summary:}

\vspace{0.3cm}

\textbf{Key Points \& Concepts:}
\begin{itemize}
  \item \textbf{Point} Description
\end{itemize}

% -------------------------- DICHEN FENG PAPER 1 --------------------------
\subsection*{3.1 \textit{Moving Object Tracking and Detection in Videos using MATLAB: A Review}}

\hspace*{\parindent}\textbf{Reviewer:} Dichen Feng

\vspace{0.3cm}

\textbf{Summary:} This paper implemented a single object tracking system using MATLAB which focusing on handling the poor linght conditions and occlusions.

\vspace{0.3cm}

\textbf{Key Points \& Concepts:}
\begin{itemize}
  \item \textbf{Color Recognition:} Use RGB and HSV color space to process video frames. HSV model is very effective on handling linghting variations.
  \item \textbf{Noise Reducition:} Uses a median filter t0 remove noise from images.
  \item \textbf{Appoxiimate Median Filter:} Extract moving object from video.
  \item \textbf{Kalman Filter:} Predicts and refines the position of the moving object in frmaes.
  \item \textbf{Dynamic Template Matching:} Adjust the object template if its shape changes during tracking.
\end{itemize}

% -------------------------- DICHEN FENG PAPER 2 --------------------------
\subsection*{3.2 \textit{Paper Name}}

\hspace*{\parindent}\textbf{Reviewer:} Dichen Feng

\vspace{0.3cm}

\textbf{Summary:} The Harris Hawks Optimization (HHO) algorithm is a nature-inspired metaheuristic that mimics the cooperative hunting strategy of Harris hawks to solve optimization problems by dynamically switching between exploration and exploitation phases.

\vspace{0.3cm}

\textbf{Key Points \& Concepts:}
\begin{itemize}
  \item \textbf{Harris Hawks Optimization (HHO):} Inspired by the cooperative hunting strategies of Harris hawks.
    \begin{itemize}
      \item \textbf{Exploration Phase:} Hawks search randomly for the target.
      \item \textbf{Transition Phase:} Adjusts strategy based on the target's energy and location.
      \item \textbf{Exploitation Phase:} Hawks converge on the prey's position through coordinated movements.
    \end{itemize}
  \item \textbf{Object Representation:} Uses color-weighted histogram in the HSV color space to describe the object's appearance. Also uses the Bhattacharyya distance for matching the object between frames.
  \item \textbf{Tracking Process:} Target initialized in the first frame using a bounding box. Then the color-weighted histogram of the object is computed as a feature descriptor in the HSV color space to make the algorithm less sensitive to lighting changes. 
  \item \textbf{Pseudo Code:}
    \begin{itemize}
      \item 1. Initialize the position of \(N\) Hawks. 
      \item 2. While \(i < \text{IterMax}\), do:
      \item 3. Evaluate fitness values of \(N\) Hawks.
      \item 4. Set \(P_{\text{rabbit}}\) as the best position of the rabbit from the best fitness value.  
      \item 5. For \(j = 1\) to \(N\), do: 
      \item 6. Update \(E\), \(E_0\), and \(J\) using Eq. (2).  
      \item 7. Update \(E\) using Eq. (3).  
      \item 8. If \(|E| \geq 1\), update Hawks position using Eq. (1). 
      \item 9. If \(|E| < 1\), then:  
      \item 10. If \(|E| \geq 0.5\) and \(r \geq 0.5\), update Hawks position using Eq. (3). 
      \item 11. Else if \(|E| < 0.5\) and \(r \geq 0.5\), update Hawks position using Eq. (5).  
      \item 12. Else if \(|E| \geq 0.5\) and \(r < 0.5\), update Hawks position using Eq. (8).  
      \item 13. Else if \(|E| < 0.5\) and \(r < 0.5\), update Hawks position using Eq. (9).  
      \item 14. Return \(P_{\text{rabbit}}\).  
    \end{itemize}
\end{itemize}

% -------------------------- HENRY PAK PAPER 1 --------------------------
\subsection*{4.1 \textit{Paper Name}}

\hspace*{\parindent}\textbf{Reviewer:} Henry Pak

\vspace{0.3cm}

\textbf{Summary:}

\vspace{0.3cm}

\textbf{Key Points \& Concepts:}
\begin{itemize}
  \item \textbf{Point} Description
\end{itemize}

% -------------------------- HENRY PAK PAPER 2 --------------------------
\subsection*{4.2 \textit{Paper Name}}

\hspace*{\parindent}\textbf{Reviewer:} Henry Pak

\vspace{0.3cm}

\textbf{Summary:}

\vspace{0.3cm}

\textbf{Key Points \& Concepts:}
\begin{itemize}
  \item \textbf{Point} Description
\end{itemize}

% -------------------------- SANTAM PAPER 1 --------------------------
\subsection*{5.1 \textit{Paper Name}}

\hspace*{\parindent}\textbf{Reviewer:} Santam

\vspace{0.3cm}

\textbf{Summary:}

\vspace{0.3cm}

\textbf{Key Points \& Concepts:}
\begin{itemize}
  \item \textbf{Point} Description
\end{itemize}

% -------------------------- SANTAM PAPER 2 --------------------------
\subsection*{5.2 \textit{Paper Name}}

\hspace*{\parindent}\textbf{Reviewer:} Santam

\vspace{0.3cm}

\textbf{Summary:}

\vspace{0.3cm}

\textbf{Key Points \& Concepts:}
\begin{itemize}
  \item \textbf{Point} Description
\end{itemize}

\end{document}
